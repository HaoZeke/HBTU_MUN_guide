\chapter{General Assembly}
\section{Introduction}
At the center of the UN system is the 193-member 
\acrfull{ga}, comprised of seven main committees and various 
subsidiary and related bodies. The \acrshort{ga} serves primarily as a forum for discussing general issues such as international peace and security and international collaboration in economic, social, cultural, educational, and health fields. The \acrshort{ga} is also able to establish committees and other bodies to study and report on specific issues. Although the decisions of the \acrshort{ga} have no binding legal force upon member-states, they do carry the weight of the moral authority of the world community.

\section{Powers}
The powers of the \acrshort{ga} include the ability to make 
recommendations on the general principle of maintaining 
international peace and security. It may discuss any question that 
is not being discussed by the Security Council and make further 
recommendations. When required, it functions to commence 
studies and make recommendations to promote international 
political cooperation, human rights and fundamental freedoms, 
the development of international law, and collaboration in 
economic, social, educational and health sectors.
\begin{tip}[GA Voting]{tip:gav}
To pass substantiative resolutions, the GA requires a two-thirds majority. \\
All other non-substantiative questions are decided by simple majority.
\end{tip}
\section{Why GA?}
With 193 seats, a full GA is one of the largest 
committees to sit in. Participation in a committee of this size 
will give you a chance to practice your public speaking skills and 
allow you to interact with a large number of delegates. Also, with 
the one member, one vote structure, the GA gives all countries a 
level ground for participation—even a small island state has just 
as much voting influence as a large and populous superpower!

\section{Committees}

\acrfull{ga} allocates items relevant to its work among its six Main Committees, which discuss them, seeking where possible to harmonize the various approaches of States, and then present to a plenary meeting of the Assembly draft resolutions and decisions for consideration.

The committees are:
\begin{itemize}
	\item \textbf{First Commitee} (Disarmament and International Security);
	\item \textbf{Second Committee} (Economic and Financial);
	\item \textbf{Third Committee} (Social, Humanitarian and Cultural);
	\item \textbf{Fourth Committee} (Special Political and Decolonization);
	\item \textbf{Fifth Committee} (Administrative and Budgetary);
	\item \textbf{Sixth Committee} (Legal).
\end{itemize}

The work of the United Nations year-round derives largely from the mandates given by
\begin{itemize}
	\item The General Assembly — that is to say, the will of the majority of the mem­bers as expressed
	\item In resolutions and decisions adopted by the Assembly. That work is carried out:
	\item By committees and other bodies established by the Assembly to study and report on
	\item Specific issues, such as disarmament, peacekeeping, development and human rights;
	\item In international conferences called for by the Assembly; and
	\item By the Secretariat of the United Nations — the Secretary-General and his staff of international civil servants.
\end{itemize}
