\chapter{Points}
Rising to points is the second of the two main methods of entering into a discussion with the Executive Board, the second being the proposal of motions.
\textbf{Do not} raise points wihout having good reason to.
\section{Points of Personal Privilege}
Whenever a delegate experiences considerable personal discomfort which impairs his/her ability to participate in the proceedings, he or she may rise to a Point of Personal Privilege to request that the discomfort be corrected. While a Point of Personal Privilege may interrupt a speaker, delegates should use this power with the 
utmost discretion.
\begin{itemize}
	\item Cannot refer to the content of a speaker's statement.
	\item Not debatable and requires no seconds.
\end{itemize}
\textbf{Example}\\
\textit{Could the speaker raise his/her as he/she is inaudible.}
\begin{note}[Format]{forma:poc}
\emph{Delegate:} \textbf{Raise Placard} \\
\emph{EB:} The \Gls{chair} recognizes \emph{Delegate}; to what point do you rise? \\
\emph{Delegate:} Honorable \Gls{chair}, Point of Personal Privilege: \emph{Query} \\
\emph{EB:} Recognized/Not recognized OR In order/Not in order. \\
\end{note}

\section{Points of Order}
During the discussion of any matter, a delegate may rise to a Point of Order to indicate an instance of improper parliamentary procedure. 
\begin{itemize}
	\item The Point of Order will be immediately decided by the \Gls{chair} in accordance with these rules of procedure.
	\item The \Gls{chair} may rule out of order those points which are dilatory or improper; such a decision is unappealable.
	\item A representative rising to a Point of Order may not speak on the substance of the matter under discussion.
	\item A Point of Order may only interrupt a speaker when the speech itself is not following proper parliamentary procedure.
\end{itemize}
\textbf{Example}\\
\textit{Is it in order for delegates to \gls{yield} the floor more than once, as the floor was just yielded to Japan?}
\begin{note}[Format]{forma:poc}
\emph{Delegate:} \textbf{Raise Placard} \\
\emph{EB:} The \Gls{chair} recognizes \emph{Delegate}; to what point do you rise? \\
\emph{Delegate:} Honorable \Gls{chair}, Point of Order: \emph{Query} \\
\emph{EB:} Recognized/Not recognized OR In order/Not in order. \\
\end{note}

\section{Points of Parliamentary Inquiry}
When the floor is open, a delegate may rise to a Point of Parliamentary Inquiry to ask the Chair a question regarding the rules of procedure.
\begin{itemize}
	\item May \textbf{never} interrupt a speaker.
	\item Delegates with substantive questions should not rise to this Point, but should rather approach the \Gls{chair} during \gls{caucus}.
\end{itemize}
\textbf{Example}\\
\textit{What is an abstention?}
\begin{note}[Format]{forma:poc}
\emph{Delegate:} \textbf{Raise Placard} \\
\emph{EB:} The \Gls{chair} recognizes \emph{Delegate}; to what point do you rise? \\
\emph{Delegate:} Honorable \Gls{chair}, Point of Parliamentary Enquiry: \emph{Query} \\
\emph{EB:} Recognized/Not recognized OR In order/Not in order. \\
\end{note}