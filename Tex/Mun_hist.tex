\chapter{Model United Nations}

\section{History}
Established in 1945 after the end of the Second World War, the 
United Nations is the largest, most comprehensive international 
organization in the world. The need for such a body as a forum 
for dialogue, maintenance of world peace, and development 
of social progress became apparent after the devastating 
consequences of the two World Wars. With the addition of 
South Sudan in July 2011, the UN now has 193 member states 
representing the vast majority of the world’s population.
The United Nations is led by the Secretary General, who oversees 
the workings of the entire organization. There are five main 
principal organs of the United Nations system:
\begin{itemize}
	\item Security Council
	\item General Assembly
	\item Economic and Social Council
	\item International Court of Justice
	\item Secretariat
\end{itemize}
In addition to these organs, there are numerous subsidiary 
organizations and committees that are focused in specialized 
areas. The more popular of these include the UN Environment 
Programme, the UN High Commissioner for Refugees, the 
World Health Organization, and the World Bank.

The UN turned out not to be the mechanism for global peace for which many had hoped; instead, the organization's true success has been in its contributions to a global political culture that demands respect between nations, discourages conflict, and 
advocates for the peaceful resolution of the conflicts that it cannot prevent. Among the philosophical underpinnings of the UN system are beliefs that all nations are sovereign and equal, that members are to fulfill in good faith the obligations that they have assumed under the UN Charter, that international disputes are to be resolved by peaceful means, and that the organization is not to intervene in matters essentially within the domestic jurisdiction of any state. As the organization has grown in size—the size of its membership has nearly quadrupled since the time of its founding—these principles of respect and amity between nations have become increasingly ingrained in nations` foreign policies. 

\section{Purposes \& Principles}
\subsection{Purposes}
Purposes and principles
The purposes of the United Nations, as set forth in the Charter, are:
\begin{itemize}
	\item \textbf{To maintain international peace and security}, and to that end: to take effective collective measures for the prevention and removal of threats to the peace, and for the suppression of acts of aggression or other breaches of the peace, and to bring about by peaceful means, and in conformity with the principles of justice and international law, adjustment or settlement of international disputes or situations which might lead to a breach of the peace;
	\item \textbf{To develop friendly relations among nations} based on respect for the principle of equal rights and self-determination of peoples, and to take other appropriate measures to strengthen universal peace;
	\item \textbf{To achieve international co-operation} in solving international problems of an economic, social, cultural, or humanitarian character, and in promoting and encouraging respect for human rights and for fundamental freedoms for all without distinction as to race, sex, language, or religion;
	\item \textbf{To be a center for harmonizing the actions of nations} in the attainment of these common ends.
\end{itemize}
\subsection{Principles}
The United Nations acts in accordance with the following principles:
\begin{itemize}
	\item It is based on the sovereign equality of all its members;
	\item All members are to fulfil in good faith their Charter obligations;
	\item They are to settle their international disputes by peaceful means and without endangering international peace and security and justice;
	\item They are to refrain from the threat or use of force against any other state;
	\item They are to give the United Nations every assistance in any action it takes in accordance with the Charter;
	\item Nothing in the Charter is to authorize the United Nations to intervene in mat­ters which are essentially within the domestic jurisdiction of any state.
\end{itemize}