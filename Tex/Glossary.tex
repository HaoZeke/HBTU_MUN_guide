\makeglossaries

\newglossaryentry{abstain}
{
        name={Abstain},
        description={During voting, a delegate may abstain, meaning that they are not voting for or against the resolution.}
}
 
\newglossaryentry{adjourn}
{
        name={Adjourn},
        description={All committee sessions end with a motion to adjourn. If majority votes to adjourn, then debate is suspended until the next meeting.}
}

\newglossaryentry{agenda}
{
        name={Agenda},
        description={Following roll call, the first duty of the committee will be to set the agenda. The agenda is the order in which topics will be discussed.}
}

\newglossaryentry{amendment}
{
	name={Amendment},
	description={An amendment is a change made to a draft resolution, and can either be ``friendly'' or ``unfriendly''. A friendly amendment pertains to a change supported by the original sponsors, and is passed automatically. An unfriendly amendment is a proposed change to a draft resolution that is not supported by all of the original sponsors, and thus, must be voted on by the whole committee.}
}

\newglossaryentry{caucus}
{
	name={Caucus},
	description={A caucus is a break from the formal debate that is the primary and secondary speakers lists, in which delegates can more easily address a given topic. A caucus can be moderated or unmoderated.}
}

\newglossaryentry{censure}
{
	name={Censure},
	description={If a delegate is behaving irresponsibly, speaking in irrelevant terms, or halting committee progress, committee members may vote to have said delegate constrained or censured from participating in debate for a given period of time.}
}

\newglossaryentry{chair}
{
	name={Chair},
	description={The Chair is a member of the dais that moderates debate, keeps time, regulates points and motions, and enforces the rules and procedures. Also known as the moderator.}
}


\newglossaryentry{committee session}
{
	name={Committee Session},
	description={Committee sessions are the times scheduled during the conference in which delegates formally meet in committees to debate their respective topics.}
}

\newglossaryentry{dias}
{
	name={Dais},
	description={Generally consisting of a Chair and a Director, the dais is the small group of staff in charge of a committee session.}
}

\newglossaryentry{decorum}
{
	name={Decorum},
	description={Decorum is the order and respect that all delegates must demonstrate when participating in a committee session. The Chair will call for decorum if he or she feels that certain committee members or the committee as a whole are being disrespectful or off task.}

\newglossaryentry{dilatory}
{
	name={Dilatory},
	description={A member of the dais may rule that a point or motion is dilatory, meaning that it is out of order or inappropriate at a given time.}
}
\newglossaryentry{dele}
{
	name={Delegate},
	description={A delegate is one who acts as a representative of a member state or as an observer in a committee at a conference.}
}

\newglossaryentry{ch}
{
	name={Chairperson},
	description={The Chairperson of a committee is a member of the dais that oversees the creation of working papers and draft resolutions, acts as an expert on the topic, ensures that decorum is maintained and that delegates accurately reflect the policy of their countries.}
}

\newglossaryentry{dr}
{
	name={Draft Resolution},
	description={A draft resolution is a document that proposes a solution to a given topic, written during a committee session. If passed by the committee, the draft resolution will become a resolution.}
}

\newglossaryentry{executive board}
{
	name={Executive Board},
	description={The judging committee consisting of the Chairman, Vice-chairs and the Special rapporteur}
}

\newglossaryentry{mems}
{
	name={Member State},
	description={A member state is a country that has ratified the Charter of the United Nations and whose application to join the UN has been accepted by the General Assembly and the Security Council. Currently, there are 193 member states of the United Nations.}
}

\newglossaryentry{modcau}
{
	name={Moderated Caucus},
	description={A moderated caucus is a type of caucus in which delegates remain seated. The Chair will call upon delegates to speak one at a time for a designated period of time.}
}

\newglossaryentry{motion}
{
	name={Motion},
	description={A motion is a request made by a delegate for the committee as a whole to do something. Some motions might involve moving into a moderated or unmoderated caucus, adjourning, or introducing a draft resolution.}
}

\newglossaryentry{observer}
	name={Observer},
	description={An observer is a state, regional/national organization, or non-governmental organization that is not a member of the UN but participates in its debates. Observers can vote on procedural matters but not substantive matters.}
}

\newglossaryentry{oc}
{
	name={Operative Clause},
	description={In a written resolution, operative clauses describe how the UN will address a problem. An operative clause begins with an action verb such as ``authorizes'', ``condemns'', and ``recommends''.}
}

\newglossaryentry{placard}
{
	name={Placard},
	description={A placard is a card on which a country’s name is printed.  A delegate may raise their placard during a committee session to signal to the Chair that he or she wishes to speak.}
}

\newglossaryentry{point}
{
	name={Point},
	description={A point may be raised by a delegate requesting information or asking permission to do something during a committee session. Examples may include a point of order, point of inquiry, or point of personal privilege.}
}

\newglossaryentry{pp}
{
	name={Position Paper},
	description={A position paper is a summary of a country’s position on a topic, written by a delegate prior to a Model UN conference.}
}

\newglossaryentry{pc}
{
	name={Preambulatory Clause},
	description={Preambulatory clauses describe previ-ous actions taken on the topic and reasons why the resolution is necessary. A preambulatory clause begins with a participle or an adjective such as ``noting'', ``concerned'', ``regretting'', or ``recalling''.}
}

\newglossaryentry{quorum}
{
	name={Quorum},
	description={Quorum is the minimum number of delegates needed to be present for a committee to meet. In the General Assem-bly, a quorum consists of one third of the members to begin debate, and a majority of members to pass a resolution. In the Security Council, no quorum exists for the body to debate, but nine members must be present to pass a resolution.}
}

\newglossaryentry{resolution}
{
	name={Resolution},
	description={ A resolution is a document that has been passed by an organ of the United Nations that aims to address a par-ticular problem or issue.}
}

\newglossaryentry{ror}
{
	name={Right of Reply},
	description={A right of reply is the right to speak in response to a previous speaker’s comment, usually when a delegate feels personally insulted by another’s speech. A right of reply generally requires a written request to the Chair.}
}


\newglossaryentry{rcall}
{
	name={Roll Call},
	description={The first order of business in a Model UN committee is roll call. When the name of each member state is called by the dais, a delegate may respond “present” or “present and voting.” A delegate responding “present and voting” may not abstain on a substantive vote.}
}

\newglossaryentry{secretariat}
{
	name={Secretariat},
	description={The most senior staff of a MUN conference.}
}

\newglossaryentry{secgen}
{
	name={Secretary General},
	description={The overseer of a MUN conference, and the head of the Secretariat who oversees the planning and execution of the conference.}
}

\newglossaryentry{signatory}
{
	name={Signatory},
	description={A signatory is a country that endorses the discussion of a draft resolution. A signatory does not need to support the given resolution, and there must be three signatories in order for a draft resolution to be approved.}
}

\newglossaryentry{sm}
{
	name={Simple Majority},
	description={A simple majority is 50\% plus one of the number of delegates in a committee.}
}

\newglossaryentry{twothree}
{
	name={Two Thirds Majority},
	description={A two-thirds vote, when unqualified, means at least two-thirds of the votes cast. This voting basis is equivalent to the number of votes in favor being at least twice the number of votes against. Abstentions are excluded in calculating a two-thirds vote.}
}

\newglossaryentry{sl}
{
	name={Speakers List},
	description={The Speakers List is a list that determines the order in which delegates will speak. Whenever a new topic is opened for discussion, the Chair will create a Speakers List by asking all delegates wishing to speak to raise their placards and calling on them one at a time. During debate, a delegate may indicate that he or she wishes to be added to the Speakers List by sending a note to the dais.}
}

\newglossaryentry{spnsor}
{
	name={Sponsor},
	description={Sponsors are the writers of a draft resolution. There must be at least two sponsors in order for a draft resolution to be approved. A friendly amendment can be made only if all sponsors agree.}
}

\newglossaryentry{umodcau}
{
	name={Unmoderated Caucus},
	description={An unmoderated caucus is a caucus in which delegates leave their seats to mingle, enabling the free exchange of ideas to an extent not possible in the formal debate of the speakers lists or even in a moderated caucus. An unmoderated caucus is beneficial for delegates to collaborate in writing working papers and draft resolutions.}
}

\newglossaryentry{wp}
{
	name={Working Paper},
	description={A working paper is a document proposing a solution to a given issue, written by delegates of a committee. A working paper is usually the precursor to a draft resolution.}
}

\newglossaryentry{veto}
{
	name={Veto},
	description={Veto power is the ability of China, France, the Russian Federation, the United Kingdom, and the United States to prevent any draft resolution in the Security Council from passing, by voting no. For councils requiring consensus (such as NATO and EU), all member states have `veto power'.}
}

\newglossaryentry{vote}
{
	name={Vote},
	description={Delegates vote to indicate whether they do or do not support a proposed action for the committee. There are two types of votes: procedural and substantive. All present delegates must vote on procedural matters and may not abstain. Procedural matters have to do with the way a committee is run, as opposed to a substantive vote, which concerns action to be taken on the topic being discussed. A delegate may abstain on a substantive vote if he/she responded to roll call with ``present''.}
}

\newglossaryentry{vb}
{
	name={Voting Bloc},
	description={A voting bloc is the period during which delegates vote on proposed amendments and draft resolutions. Nobody may enter or leave the room during voting bloc.}
}

\newglossaryentry{yield}
{
	name={yield},
	description={When a delegate concludes their turn on the Speakers List, he or she decides what to do with the remaining time by yielding to another delegate, to questions, or to the Chair. Yields must be declared at the end of each speech from the Speakers List.}
}

\newacronym{ga}{GA}{General Assembly}
\newacronym{specpol}{SPECPOL}{Special Political and Decolonization Committee}
\newacronym{disec}{DISEC}{Disarmament and International Security Committee}
\newacronym{ecofin}{ECOFIN}{Economic and Financial Affairs Council}
\newacronym{sochum}{SOCHUM}{Social, Humanitarian, and Cultural Committee}
\newacronym{unsc}{UN-SC}{United Nations Security Council}