\chapter{Fourth Committee}
Here at the HBTU MUN 2017, delegates are to simulate the fourth committee of the \acrfull{ga}; the \acrfull{specpol}.

\section{History}
The Fourth Committee of the General Assembly of the United Nations is the \acrfull{specpol}. Established in 1993, \acrshort{specpol} is the combination of the Decolonization Committee (formerly the Fourth Committee) and the Special Political Committee. This committee’s inception took place in 1990 when the United Nations established 1990-2000 as the ``International Decade for the Eradication of Colonialism''. This was particularly important considering at the time of the United Nation’s creation, 750 million people lived in what would be considered to be a colonized territory. Over eighty former colonies have become independent since 1945. Today, in part due to the work of the Fourth Committee, this number has drastically decreased to approximately two million, an amount SPECPOL is still determined to address.

\section{Mandate}
\acrfull{specpol}, has a somewhat more fragmented mandate 
than other committees of the General Assembly, such as \acrshort{disec}, \acrshort{ecofin}, or \acrshort{sochum}. The UN itself describes the committee as concerning itself with a variety of subjects which include those related to:
\begin{itemize}
	\item Decolonization
	\item Palestinian refugees and human rights
	\item Peacekeeping
	\item Mine action
	\item Outer space
	\item Public information
	\item Atomic radiation 
\end{itemize}

In short, SPECPOL covers both the issue of decolonisation, as suggested by its full name, as well as any other political issues not directly dealt with by the mandates of any other UN General Assembly committee.

\begin{note}[Resolution Scope]{note:specpolsco}
Resolutions are \textbf{non-binding} and therefore peacekeeping operations or other punitive measures can \textbf{only} be suggested or recommended to the \acrfull{unsc}.
\end{note}

\section{Powers}
While \acrshort{specpol} was derived from the Disarmament and 
International Security Committee, it takes on issues that the First Committee does not address, as 
well as looking at topics with a wider scope.
Unlike other UN committees, \acrshort{specpol}
shines a spotlight on issues pertaining to occ
upation, colonization, and subjugation, with the primary goal of making all countries independent and self
sufficient from outside powers.

\begin{note}[Who Cares?]{note:specpolscare}
Resolutions are not binding, but they are very appealing to the international community, for the reason that they reflect the opinion of the majority regarding substantive matters.
\end{note}
\section{Why SPECPOL?}
The Special Political and Decolonization committee holds a very unique role in the United Nations. 
While other main committees aim to resolve current global issues, \acrshort{specpol} concerns itself with 
healing countries from the lasting impact of their troubling histories. The committee’s foremost goal is to ensure that all countries enjoy the benefits of the independence to w
hich they are entitled, and only when all countries are economically, culturally, and socially liberated can the world move forward.


Currently, SPECPOL can be viewed as the UN Security Council’s entrance door, since the issues discussed at the Council are often assessed earlier by the General Assembly through SPECPOL not only because of the broader approach it has in terms of international security, but also because it allows all UN member states to be heard before the question reaches other organs. 
SPECPOL’s 